\documentclass[Lau, oneside]{sapthesis}%remove "english" for a thesis written in Italian
%Bachelor's (laurea triennale) thesis : Lau 
%Master's (laurea specialistica) thesis: LaM 
%PhD's thesis: PhD 
\usepackage[italian]{babel} %use this package for a thesis written in Italian
\usepackage[utf8]{inputenx}
\usepackage{indentfirst}
\usepackage{microtype}
%\usepackage{chemformula}
%\usepackage{setspace}
%\usepackage{yfonts,color}https://www.overleaf.com/project/5e4187442ef81c00013a37ea
%\usepackage{siunitx}
%\usepackage{comment}
%\usepackage{multirow}
%\usepackage{varioref}
%\usepackage[bottom]{footmisc}
%\usepackage{wrapfig}
%\usepackage{float}
%\usepackage{type1cm}
\usepackage{lettrine}
\linespread{0.9}
%\usepackage{chngcntr}
\usepackage[nottoc, notlof, notlot]{tocbibind}
%\onehalfspacing
%\counterwithout{footnote}{chapter}
\usepackage{hyperref}
\hypersetup{
			hyperfootnotes=true,			
			bookmarks=true,			
			colorlinks=true,
			linkcolor=red,
                        linktoc=page,
			anchorcolor=black,
			citecolor=red,
			urlcolor=blue,
			pdftitle={A sample Bachelor's thesis for Sapienza Università di Roma},
			pdfauthor={FirstName LastName},
			pdfkeywords={thesis, sapienza, roma, university}
 }

\title{Sviluppo App}
\author{Edoardo Gabrielli}
\IDnumber{1693726}
\course[]{Informatica}
\courseorganizer{Facolt\`a di Ingegneria dell’Informazione, Informatica e Statistica}
\submitdate{2019/2020}
\copyyear{2020}
\advisor{Prof. Emanuele Panizzi}
\authoremail{gabrielli.1693726@studenti.uniroma1.it}
\examdate{23 marzo 2020}
\examiner{Prof.}

%we refer to http://ctan.mirrorcatalogs.com/macros/latex/contrib/sapthesis/sapthesis-doc.pdf for an exhaustive description of the sapthesis documentclass.


\begin{document}

\frontmatter
\maketitle
\begin{abstract}
Oggetto della tesi è lo sviluppo delle app InfoStud e InfoProf in cui mi sono occupato di risolvere i problemi pre-esistenti e aggiungere nuove funzionalità di entità variabile.
Il documento è diviso in tre macro aree: un'introduzione generale alle applicazioni, al processo di sviluppo e agli strumenti utilizzati, poi l'iniziale attività di bug-fixing che comprende anche l'aggiunta di piccole features e infine una parte dedicata allo studio, progettazione ed implementazione dell'apertura di un verbale d'esame.
\end{abstract}

\tableofcontents

\mainmatter
\chapter{Itroduzione}
\label{ch:1}
\section{InfoStud e InfoProf}
\label{sec:pres}
L'app InfoStud nasce \textit{n} anni fa dalla necessità degli studenti di avere il sistema InfoStud sviluppato da InfoSapienza su mobile.
Data la natura poco \textit{mobile-friendly} del sistema web, nacque l'app ufficiale sotto il brand di SapienzaApps.

SapienzaApps, sotto il coordinamento del Prof. Emanuele Panizzi, pubblica e mantiene progetti come SeismoCloud e GeneroCity all'interno
del Gamification Lab.

InfoStud è l'app che si rivolge unicamente agli studenti iscritti alla Sapienza e mette a disposizione un parco di funzionalità ampio
che copre le funzioni standard del sistema padre (come visualizzazione e gestione esami) e andando oltre in alcuni casi: il sistema
di gestione dell'orario didattico semi-automatico, la compilazione dei bollettini automatica, la prenotazione dei posti in biblioteca, ecc.

InfoProf, al contrario, è l'app che si rivolge ai professori della Sapienza e vuole essere uno strumento alternativo, e più intuitivo, 
del sistema web. La semplicità di utilizzo dell'app va però analizzata e progettata in modo empirico. Questo studio porta via molto tempo 
al lavoro ed è per questo che gran parte della progettazione si traduce in prototyping e test di usabilità, con eventualmente molteplici
iterazioni. Lo stato dell'arte attualmente è un'app che permette di verbalizzare gli studenti, attivare gli OPIS e cercare le aule.
Queste funzionalità hanno un denominatore in comune: sono casi d'uso in cui l'utente potrebbe non avere il computer quando ha bisogno
di utilizzarle. La verbalizzazione infatti può essere fatta subito dopo un orale, senza dover aspettare di tornare in ufficio e 
registrare i voti di tutti gli studenti. L'OPIS, date le ultime disposizioni, deve essere attivato durante la lezione. Ma un utente
potrebbe non voler portare un computer solo per attivare il codice OPIS.

%scrivere meglio questo paragrafo per ricollegarsi a quello precedente
Lo studio che io ho fatto invece riguarda un'altra funzionalità direi fondamentale: l'apertura di un verbale d'esame. Diciamo che
questa è una necessità che deriva più dal fatto che l'interfaccia web è poco \textit{user-friendly}, ma ne parlerò più approfonditamente
in \ref{sec:why}.


\section{Perché?}
\label{sec:why}


\chapter{Metodologie di sviluppo}
\label{ch:2}
\section{Organizzazione del team}
\label{sec:team}
\section{Tecnologie utilizzate}
\label{sec:tech}


\chapter{Progettazione}
\label{ch:3}
\section{Need-finding}
\label{sec:nf}
\section{Progettazione della UI}
\label{sec:ui}
\section{Implementazione}
\label{sec:dev}


\chapter{Concluzioni}
\label{ch:4}

\backmatter
\phantomsection
\begin{thebibliography}{17}

\bibitem{ref:vph}
Blanche P.A., Gailly P., et al., “\textit{Volume phase holographic gratings: large size and high diffraction efficiency}“, Optical Engineering, Vol. 43, No.11, November 2004

\bibitem{ref:science}
Cirasuolo M., et al., \textit{"MOONS Science Report"}, MOONS Document Number: VLT-TRE-MON-14620-0001, Issue: $1.0$, $31^{\textup{st}}$ January $2013$

\bibitem{ref:eso}
European Southern Observatory, \url{http://www.eso.org}

\end{thebibliography}

\end{document}