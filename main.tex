\documentclass[Lau, oneside]{sapthesis}%remove "english" for a thesis written in Italian
%Bachelor's (laurea triennale) thesis : Lau 
%Master's (laurea specialistica) thesis: LaM 
%PhD's thesis: PhD 
\usepackage[italian]{babel} %use this package for a thesis written in Italian
\usepackage[utf8]{inputenx}
\usepackage{indentfirst}
\usepackage{microtype}
%\usepackage{chemformula}
%\usepackage{setspace}
%\usepackage{yfonts,color}https://www.overleaf.com/project/5e4187442ef81c00013a37ea
%\usepackage{siunitx}
%\usepackage{comment}
%\usepackage{multirow}
%\usepackage{varioref}
%\usepackage[bottom]{footmisc}
%\usepackage{wrapfig}
%\usepackage{float}
%\usepackage{type1cm}
\usepackage{lettrine}
\linespread{0.9}
%\usepackage{chngcntr}
\usepackage[nottoc, notlof, notlot]{tocbibind}
%\onehalfspacing
%\counterwithout{footnote}{chapter}
\usepackage{hyperref}
\hypersetup{
			hyperfootnotes=true,			
			bookmarks=true,			
			colorlinks=true,
			linkcolor=red,
                        linktoc=page,
			anchorcolor=black,
			citecolor=red,
			urlcolor=blue,
			pdftitle={A sample Bachelor's thesis for Sapienza Università di Roma},
			pdfauthor={FirstName LastName},
			pdfkeywords={thesis, sapienza, roma, university}
 }

\title{Sviluppo App}
\author{Edoardo Gabrielli}
\IDnumber{1693726}
\course[]{Informatica}
\courseorganizer{Facolt\`a di Ingegneria dell’Informazione, Informatica e Statistica}
\submitdate{2019/2020}
\copyyear{2020}
\advisor{Prof. Emanuele Panizzi}
\authoremail{gabrielli.1693726@studenti.uniroma1.it}
\examdate{}
\examiner{}

%we refer to http://ctan.mirrorcatalogs.com/macros/latex/contrib/sapthesis/sapthesis-doc.pdf for an exhaustive description of the sapthesis documentclass.


\begin{document}

\frontmatter
\maketitle

\begin{abstract}
Oggetto della tesi è lo sviluppo delle app InfoStud e InfoProf in cui mi sono occupato di risolvere i problemi pre-esistenti e aggiungere nuove funzionalità di entità variabile.
Il documento è diviso in tre macro aree: un'introduzione generale alle applicazioni, al processo di sviluppo e agli strumenti utilizzati, poi l'iniziale attività di bug-fixing che comprende anche l'aggiunta di piccole features e infine una parte dedicata allo studio, progettazione ed implementazione dell'apertura di un verbale d'esame.
\end{abstract}

\tableofcontents

\mainmatter
\chapter{Itroduzione ai progetti}
\section{InfoStud}
InfoStud è l'app che si interfaccia 

\chapter{Near-infrared multi-object spectroscopy}
\label{chap:1} 
\section{Scientific Case}
\label{sec:caso}
Over the last decades, innovative observational techniques have been developed to allow spectrographs observing...

\chapter{MOONS}
\label{chap:2}
\section{The Very Large Telescope}
Property of the European Southern Observatory...

\section{The Multi-Object Optical and Near-infrared Spectrograph}
\label{sec:moons}

The \textit{Multi-Object Optical and Near-infrared Spectrograph} is a future generation MOS instrument for the VLT. 

\chapter{Conclusions}
The grasping power of the mirror..

\backmatter
\phantomsection
\begin{thebibliography}{17}

\bibitem{ref:vph}
Blanche P.A., Gailly P., et al., “\textit{Volume phase holographic gratings: large size and high diffraction efficiency}“, Optical Engineering, Vol. 43, No.11, November 2004

\bibitem{ref:science}
Cirasuolo M., et al., \textit{"MOONS Science Report"}, MOONS Document Number: VLT-TRE-MON-14620-0001, Issue: $1.0$, $31^{\textup{st}}$ January $2013$

\bibitem{ref:eso}
European Southern Observatory, \url{http://www.eso.org}

\end{thebibliography}

\end{document}