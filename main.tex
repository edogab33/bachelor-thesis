\documentclass[Lau, oneside]{sapthesis}%remove "english" for a thesis written in Italian
%Bachelor's (laurea triennale) thesis : Lau 
%Master's (laurea specialistica) thesis: LaM 
%PhD's thesis: PhD 
\usepackage[italian]{babel} %use this package for a thesis written in Italian
\usepackage[utf8]{inputenx}
\usepackage{indentfirst}
\usepackage{microtype}
%\usepackage{chemformula}
%\usepackage{setspace}
%\usepackage{yfonts,color}https://www.overleaf.com/project/5e4187442ef81c00013a37ea
%\usepackage{siunitx}
%\usepackage{comment}
%\usepackage{multirow}
%\usepackage{varioref}
%\usepackage[bottom]{footmisc}
%\usepackage{wrapfig}
%\usepackage{float}
%\usepackage{type1cm}
\usepackage{lettrine}
%portare linespread a 1.1-1.3
\linespread{1.1}
%\usepackage{chngcntr}
\usepackage[nottoc, notlof, notlot]{tocbibind}
%\onehalfspacing
%\counterwithout{footnote}{chapter}
\usepackage{hyperref}
\hypersetup{
			hyperfootnotes=true,			
			bookmarks=true,			
			colorlinks=true,
			linkcolor=red,
                        linktoc=page,
			anchorcolor=black,
			citecolor=red,
			urlcolor=blue,
			pdftitle={A sample Bachelor's thesis for Sapienza Università di Roma},
			pdfauthor={FirstName LastName},
			pdfkeywords={thesis, sapienza, roma, university}
 }

\title{Sviluppo App}
\author{Edoardo Gabrielli}
\IDnumber{1693726}
\course[]{Informatica}
\courseorganizer{Facolt\`a di Ingegneria dell’Informazione, Informatica e Statistica}
\submitdate{2019/2020}
\copyyear{2020}
\advisor{Prof. Emanuele Panizzi}
\authoremail{gabrielli.1693726@studenti.uniroma1.it}
\examdate{23 marzo 2020}
\examiner{Prof.}

%we refer to http://ctan.mirrorcatalogs.com/macros/latex/contrib/sapthesis/sapthesis-doc.pdf for an exhaustive description of the sapthesis documentclass.


\begin{document}

\frontmatter
\maketitle
\begin{abstract}
Oggetto della tesi è lo sviluppo delle app InfoStud e InfoProf in cui mi sono occupato di risolvere i problemi pre-esistenti e aggiungere nuove funzionalità di entità variabile.
Il documento è diviso in tre macro aree: un'introduzione generale alle applicazioni, al processo di sviluppo e agli strumenti utilizzati, poi l'iniziale attività di bug-fixing che comprende anche l'aggiunta di piccole features e infine una parte dedicata allo studio, progettazione ed implementazione dell'apertura di un verbale d'esame.
\end{abstract}

\tableofcontents

\mainmatter
\chapter{Itroduzione}
\label{ch:1}
\section{InfoStud e InfoProf}
\label{sec:pres}
L'app InfoStud nasce \textit{n} anni fa dalla necessità degli studenti di avere il sistema InfoStud sviluppato da InfoSapienza su mobile.
Data la natura poco \textit{mobile-friendly} del sistema web, nacque l'app ufficiale sotto il brand di SapienzaApps.

SapienzaApps, sotto il coordinamento del Prof. Emanuele Panizzi, pubblica e mantiene progetti come SeismoCloud e GeneroCity all'interno
del Gamification Lab.

InfoStud è l'app che si rivolge unicamente agli studenti iscritti alla Sapienza e mette a disposizione un parco di funzionalità ampio
che copre le funzioni standard del sistema padre (come visualizzazione e gestione esami) e andando oltre in alcuni casi: il sistema
di gestione dell'orario didattico semi-automatico, la compilazione dei bollettini automatica, la prenotazione dei posti in biblioteca, ecc.

InfoProf, al contrario, è l'app che si rivolge ai professori della Sapienza e vuole essere uno strumento alternativo, e più intuitivo, 
del sistema web. La semplicità di utilizzo dell'app va però analizzata e progettata in modo empirico. Questo studio porta via molto tempo 
al lavoro ed è per questo che gran parte della progettazione si traduce in prototyping e test di usabilità, con eventualmente molteplici
iterazioni. Lo stato dell'arte attualmente è un'app che permette di verbalizzare gli studenti, attivare gli OPIS e cercare le aule.
Queste funzionalità hanno un denominatore in comune: sono casi d'uso in cui l'utente potrebbe non avere il computer quando ha bisogno
di utilizzarle. La verbalizzazione infatti può essere fatta subito dopo un orale, senza dover aspettare di tornare in ufficio e 
registrare i voti di tutti gli studenti. L'OPIS, date le ultime disposizioni, deve essere attivato durante la lezione. Ma un utente
potrebbe non voler portare un computer solo per attivare il codice OPIS.

%scrivere meglio questo paragrafo per ricollegarsi a quello precedente
Lo studio che io ho fatto invece riguarda un'altra funzionalità direi fondamentale: l'apertura di un verbale d'esame. Diciamo che
questa è una necessità che deriva più dal fatto che l'interfaccia web è poco \textit{user-friendly}, ma ne parlerò più approfonditamente
in \ref{sec:why} e in \ref{sec:nf}.


\section{Perché?}
\label{sec:why}
%trovare articoli scientifici in merito. 
Lo smartphone è ormai uno strumento radicato nella nostra quotidianeità. Sono innumerevoli i modi in cui ha cambiato la nostra vita e la
produttività è uno degli aspetti centrali. Nel settore pubblico l'evoluzione dei sistemi informatici però pecca di una rigorosa analisi 
e progettazione, il che comporta perdite di tempo e difficoltà nell'utilizzo.
Dall'indagine preliminare che ho svolto infatti emerge che alcuni intervistati trovano l'interfaccia web \textit{error-prone}, mentre 
un professore assunto recentemente, ha dichiarato che la piattaforma non è affatto user-friendly e che la logica vorrebbe che ci fosse
un tasto "crea appello", non che debba andarlo a cercare in "verbalizzazione". Anche all'interno del team di sviluppo, quando ci è
stata presentata per la prima volta l'interfaccia, abbiamo faticato a capire come svolgere i task che un professore è tenuto normalmente
a fare. Tolti errori grossolani come l'assegnazione di termini differenti alla stessa cosa [trovare un termine più adatto] 
\textbf{(inserire figura)}, esistono delle criticità che vanno contro la logica comune: se andiamo ad osservare la lista degli insegnamenti,
alla destra di ogni elemento possiamo osservare un numero che rappresenta il numero di \textit{corsi} selezionati accoppiati con quell'
insegnamento. Se selezioniamo il checkbox dell'elemento e andiamo all'interno del sotto-menù, deselezionando tutti i corsi, l'insegnamento
rimarrà selezionato. Una buona interfaccia \textbf{(citare)}, invece, dovrebbe aiutare l'utente ad evitare gli errori.

%citare le statistice del miur riguardo età e numero di docenti della sapienza

%L'approccio che ho avuto quindi è stato quello di capire cosa gli utenti ritengono più importante nell'apertura di un verbale e mettere
%in secondo piano le funzionalità che vengono usate raramente. Vedremo infatti che paradossalmente l'interfaccia web, ai docenti che 
%hanno imparato ad usarla, piace.


\chapter{Metodologie di sviluppo}
\label{ch:2}
\section{Organizzazione}
\label{sec:team}
Il team è composto da studenti impiegati nello sviluppo per circa tre mesi. E' prassi iniziare ad ambientarsi nei progetti attraverso
l'attività di bug-fixing. Familiarizzare con il codice e con i colleghi è centrale dato che per molti di noi è la prima volta che 
ci troviamo dentro un progetto di discrete dimensioni e all'interno di un contesto semi-professionale.

Dato che molti di noi si fermano solo per tre mesi, la metodologia di sviluppo che si adatta meglio alle nostre esigenze è l'Agile.
Lo sviluppo agile si basa su un insieme di regole, citandone due significative:
%centrare le citazioni
"Our highest priority is to satisfy the customer
through early and continuous delivery of valuable software."

"Deliver working software frequently, from a
couple of weeks to a couple of months, with a
preference to the shorter timescale."

Ciò detto si collega ad un altro aspetto dello sviluppo software del Gamification Lab, quello del Continuous Integration e Continuous
Delivery. Per lavorare utilizziamo quotidianamente GitLab, che è un sistema di \textit{controllo versione}. Questo aiuta a tenere
traccia delle modifiche apportate al codice, tornare indietro, pubblicare issue e assegnarle. Questo aiuta a mantenere lo sviluppo
\textit{agile}, cioè aggiungere piccole funzionalità nel giro di una settimana e mantenerne la qualità attraverso le code review, per
poi integrarle nel programma beta e infine nell'app ufficiale.



\section{Tecnologie utilizzate}
\label{sec:tech}


\chapter{Progettazione}
\label{ch:3}
%inserire diagramma di flusso di come viene fatta la progettazione (nf -> protoryping -> test -> repeat)
\section{Need-finding}
\label{sec:nf}
\section{Progettazione della UI}
\label{sec:ui}
%citare "Developing SMASH"
\section{Implementazione}
\label{sec:dev}
%inserire qualche snippet e magari parlare dei limiti tecnologici di ionic e come li ho superati


\chapter{Concluzioni}
\label{ch:4}

\backmatter
\phantomsection
\begin{thebibliography}{17}

\bibitem{ref:vph}
Blanche P.A., Gailly P., et al., “\textit{Volume phase holographic gratings: large size and high diffraction efficiency}“, Optical Engineering, Vol. 43, No.11, November 2004

\bibitem{ref:science}
Cirasuolo M., et al., \textit{"MOONS Science Report"}, MOONS Document Number: VLT-TRE-MON-14620-0001, Issue: $1.0$, $31^{\textup{st}}$ January $2013$

\bibitem{ref:eso}
European Southern Observatory, \url{http://www.eso.org}

\end{thebibliography}

\end{document}